\section*{Discussion of Results and Error Analysis}

\noindent
It is easy to tell that none of the collisions are perfectly elastic or inelastic by the difference in $p$ and $p'$ 
(and similarly, the difference between $KE$ and $KE'$). The experimental errors are likely the result of an imperfectly 
leveled air track, imperfections in the track, and friction between the gliders and track. While the percent differences 
are high for each trial, the difference in $p$ and $p'$ can be attributed to these factors, because we know that the 
momentum is conserved overall.

\bigskip

\noindent
We can see from the calculations that the inelastic collisions had lower percent differences across the board.

\bigskip

\noindent
The percent difference in $KE$ and $KE'$ in the elastic collision can be attributed to the conversion of kinetic energy 
into other forms. Rubber bands convert kinetic energy into heat energy when stretched, and we found that the rubber bands 
did not fully prevent the metal parts of the gliders from colliding and producing sound. The energy is still conserved, 
however: even though the calculations show that the amount of kinetic energy in the system decreased after the collision, 
the energy has only been lost to the system.

\bigskip

\noindent
The percent difference in $KE$ and $KE'$ in the inelastic collision can be explained similarly. While the system 
loses less energy to heat and sound than in an elastic collision, energy is used to 