\section*{Discussion of Results and Error Analysis}

It is easy to tell that none of the collisions are perfectly elastic or inelastic by the difference in $p$ and $p'$ (and similarly, the difference between $KE$ and $KE'$). The experimental errors are likely the result of an imperfectly leveled air track, imperfections in the track, and friction between the gliders and track. While the percent differences are high for each trial, the difference in $p$ and $p'$ can be attributed to these factors, because we know that the momentum is conserved overall.

\bigskip

Aside from the large margin of error in trial 1, we can see from the calculations that the inelastic collisions had lower percent differences on average. This is likely due to less conversion of kinetic energy to other forms. In the elastic collisions, kinetic energy was converted to heat in the rubber bands as they stretched, and the gliders transferred quite a bit of momentum downward into the track as they collided a bit too hard and jumped a small distance above the track on the collision. This shows that the energy is still conserved: even though the calculations show that the amount of kinetic energy in the system decreased after the collision, the energy is merely transferred out of the system.

\bigskip

The percent difference in $KE$ and $KE'$ in the inelastic collision can be explained similarly. While the system loses less energy to heat and sound than in an elastic collision, energy is used to attach the gliders as the nail sinks into the wax tube and sticks. Also similarly to in the elastic collisions, the collision would cause the gliders to rise a short distance above the track and crash back down.

\bigskip

If we were to suppose that the air track was tilted during the experiments, momentum would still be conserved. Since a non-zero angle of inclination in the air track would introduce a $y$ component to the calculations (and a constant force in the negative $y$ direction), it would appear that momentum is lost. However, this difference in momentum would be attributed to gravity, which is a conservative force. Because of this, we know that momentum would still be conserved in the system, regardless of the difference in final values.
