\section*{Introduction}

The purpose of this experiment is to gain the ability to define momentum and kinetic energy, and to be able to explain conservation of momentum. By using the law of conservation of momentum, we will predict the resulting speed and direction of each object involved in a one-dimensional collision. We will study the differences between elastic and inelastic collisions and compare initial and final kinetic energies in these imperfect collisions.

\bigskip

Momentum, like energy, is always conserved. While energy can leave a system (e.g. kinetic energy converted to heat or sound), it is still conserved overall. In the scenario that we replicate in this experiment, a moving glider transfers momentum to a glider at rest. We can observe the conservation of momentum in a system by calculating the total momentum of the system before ($p$) and after ($p'$) a collision:
\[ p = m_{1}v_{1} + m_{2}v_{2} \]
\[ p' = m_{1}v_{1}' + m_{2}v_{2}' \]
where $v_i$ is the velocity of the $i$-th glider after the collision. Because momentum is conserved, we expect to find that $p=p'$ within a reasonable margin of experimental error. Conservation of energy, while not the focus of this experiment, can be observed in a similar manner:
\[ KE = \frac{1}{2} mv_{i}^{2} \]
\[ KE' = \frac{1}{2} mv_{f}^{2} \]
Again, we expect that $KE=KE'$ within a reasonable margin of experimental error, and we expect that some kinetic energy will be converted to other forms and leave the system.

\bigskip

We will observe the conservation of momentum in a linear, two-object system of two gliders on an air track. The air track will provide a surface with a very low coefficient of friction, in order to avoid loss of kinetic energy to the nonconservative friction force. The gliders will be adorned with accessories to help facilitate elastic and inelastic collisions, so that we can observe differences in these types of collisions.
