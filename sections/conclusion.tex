\section*{Conclusion}

Our experiment allowed us to observe and study the conservation of momentum in a one-dimensional, two-object system. We observed the differences between elastic and inelastic collisions and gained experience calculating values for momentum and kinetic energy using the formulas given previously, and found that our results support the theory of conservation of momentum.

\bigskip

This experiment could be improved by ensuring that the air track is completely level and free of imperfections. Ensuring accurate hand-measurement of values such as the mass of the gliders would also increase the accuracy of the calculations, which in turn would reduce experimental error. In addition, applying too large of a push force to glider 1 toward glider 2 resulted in a larger difference between initial and final values as the gliders crashed into the track and bled kinetic energy out of the system.

\bigskip

This method of measuring $m$ and $v$ and using the values to calculate $p$ and $KE$ is sufficiently accurate. The errors incurred throughout the course of our experiment were largely preventable, and do not reflect an inaccuracy in the method. The method is as precise as the measurements of mass and velocity allow, and since the photo gates and computer provide quite a precise measurement of velocity over a known distance, the method proves to be precise as well.
