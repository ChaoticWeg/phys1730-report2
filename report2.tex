\documentclass[12pt,letterpaper,titlepage]{report}

% packages

\usepackage[final]{pdfpages}
\usepackage[utf8]{inputenc}
\usepackage{mathptmx}
\usepackage{geometry}
\usepackage{float}
\usepackage{subcaption}
\usepackage{multirow}

% variables

\newcommand{\myTitle}{Conservation of Momentum}
\newcommand{\myName}{Shawn Lutch}
\newcommand{\myPeriod}{PHYS 1730.506}

% metadata

\usepackage[pdftex,
            pdfauthor={\myName{}},
            pdftitle={\myTitle{}},
            pdfsubject={\myPeriod{}},
            pdfproducer={LaTeX},
            pdfcreator={latexpdf}]{hyperref}

% layout

\geometry{
    letterpaper,
    lmargin=0.75in,
    rmargin=0.75in,
    tmargin=1.0in,
    bmargin=1.0in
}

% the real fun begins

\begin{document}

% TITLE PAGE

\title{\myTitle{}}
\author{\myName{}\\ \myPeriod{}}
\date{\today}
\maketitle


% ABSTRACT

\section*{Abstract}

\noindent
The purpose of this lab is to study conservation of momentum and the differences between elastic 
and inelastic collisions. This lab focuses on collisions in a linear, two-object system. By sliding 
two gliders with known masses toward each other on an air track and calculating the total momentum 
of the system ($p$) and after ($p'$) the collision, we see that $p = p'$, and thus momentum is 
conserved.

% INTRODUCTION

\section*{Introduction}

\noindent


% APPARATUS

\section*{Apparatus}

\begin{itemize}
    \item Air track
    \item Two gliders
    \item Two photo gates
    \item Computer with timing and analysis software
\end{itemize}

% PROCEDURE

\section*{Procedure}

\noindent
The air track and compressor were set up by the lab instructors prior to
our experiment. We ensured that the track was level by adjusting the base
support screws until a glider in the middle of the track remained stationary.
We then placed a photogate 30 cm from each end of the track and placed a 
rubber band bumper on each end of the track, to prevent the gliders from
colliding harshly into the ends of the track after colliding with one another.

% DATA

\section*{Data}

\subsection*{Glider Masses}

\begin{tabular}{ | c | c c | } \hline
    Trial & $m_1$ & $m_2$ \\ \hline
    1 & 191.19g & 191.81g \\
    2 & 312.70g & 191.81g \\
    3 & 191.19g & 191.81g \\
    4 & 312.70g & 191.81g \\ \hline
\end{tabular}


% CALCS AND GRAPHS

\section*{Calculations and Graphs}
\captionsetup[subfigure]{labelformat=empty}

\subsection*{Inelastic Collisions}

\begin{figure}[H]
    \centering
    \begin{subfigure}[b]{0.49\textwidth}
        \centering
        \caption{Momentum -- Inelastic}
        \begin{tabular}{ | c | c c | c | c | } \hline
    Trial & $p$ & $p'$ & \% Diff. & Av. \% Diff. \\ \hline
    1 & 0.149 & 0.092 & 47.3 & \multirow{3}{*}{24.9} \\
    1 & 0.113 & 0.100 & 12.2 & \\
    1 & 0.134 & 0.115 & 15.3 & \\ \hline
    2 & 0.172 & 0.166 & 3.6 & \multirow{3}{*}{10.8} \\
    2 & 0.188 & 0.187 & 0.5 & \\
    2 & 0.181 & 0.136 & 28.4 & \\ \hline
\end{tabular}

    \end{subfigure}
    \begin{subfigure}[b]{0.49\textwidth}
        \centering
        \caption{Kinetic Energy -- Inelastic}
        \begin{tabular}{ | c | c c | c | c | } \hline
    Trial & $KE$ & $KE'$ & \% Diff. & Av. \% Diff. \\ \hline
    1 & 0.058 & 0.011 & 136.2 & \multirow{3}{*}{106.6} \\
    1 & 0.033 & 0.013 & 89.9 & \\
    1 & 0.047 & 0.017 & 93.8 & \\ \hline
    2 & 0.047 & 0.027 & 54.1 & \multirow{3}{*}{66.3} \\
    2 & 0.056 & 0.035 & 46.2 & \\
    2 & 0.053 & 0.018 & 98.5 & \\ \hline
\end{tabular}

    \end{subfigure}
\end{figure}

\subsection*{Elastic Collisions}

\begin{figure}[H]
    \centering
    \begin{subfigure}[b]{0.49\textwidth}
        \centering
        \caption{Momentum -- Elastic}
        \begin{tabular}{ | c | c c | c | c | } \hline
    Trial & $p$ & $p'$ & \% Diff. & Av. \% Diff. \\ \hline
    3 & 0.098 & 0.063 & 43.5 & \multirow{3}{*}{32.6} \\
    3 & 0.099 & 0.077 & 25.0 & \\
    3 & 0.098 & 0.073 & 29.2 & \\ \hline
    4 & 0.159 & 0.092 & 53.4 & \multirow{3}{*}{60.4} \\
    4 & 0.169 & 0.090 & 61.0 & \\
    4 & 0.200 & 0.100 & 66.7 & \\ \hline
\end{tabular}

    \end{subfigure}
    \begin{subfigure}[b]{0.49\textwidth}
        \centering
        \caption{Kinetic Energy -- Elastic}
        \begin{tabular}{ | c | c c | c | c | } \hline
    Trial & $KE$ & $KE'$ & \% Diff. & Av. \% Diff. \\ \hline
    3 & 0.049 & 0.032 & 41.9 & \multirow{3}{*}{33.2} \\
    3 & 0.050 & 0.038 & 27.2 & \\
    3 & 0.049 & 0.036 & 30.5 & \\ \hline
    4 & 0.041 & 0.025 & 48.5 & \multirow{3}{*}{48.6} \\
    4 & 0.046 & 0.028 & 48.6 & \\
    4 & 0.064 & 0.039 & 48.5 & \\ \hline
\end{tabular}

    \end{subfigure}
\end{figure}
    

% DISCUSSION

\section*{Discussion of Results and Error Analysis}

% CONCLUSION

\section*{Conclusion}



\end{document}
